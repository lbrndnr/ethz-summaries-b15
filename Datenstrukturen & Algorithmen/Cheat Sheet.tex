\documentclass[11pt]{article}

\usepackage{amsmath}
\usepackage{amssymb}
\usepackage{array}
\usepackage{geometry}
\usepackage{enumitem}
\usepackage{float}
\usepackage{cancel}
\usepackage{graphicx}
\usepackage[labelformat=empty]{caption}

\geometry{
	a4paper,
 	left=20mm,
 	top=20mm,
}

\setlength{\parindent}{0pt}

\begin{document}

\section{O-Notation}

\begin{equation*}
	O(c) < O(\log \log n) < O(\log n) < O((\log n)^c) <  O(\sqrt[c]{n}) < O(n \log n) = O(\log n!) < O(n^c) < O(c^n) < O(n!)
\end{equation*}

mit $c > 1$ als positive Konstante.

\emph{Tipps:}
\begin{itemize}[noitemsep]
	\item $\binom{n}{c} \in \Theta(n^c)$
	\item $n^c$ ist polynomial
\end{itemize}

\section{Hashing}

Eine Hashingfunktion sollte

\begin{itemize}[noitemsep]
	\item möglichst schnell zu berechnen sein
	\item Datensätze möglichst gleichmäßig auf den Speicherbereich verteilen
	\item Häufungen (cluster) fast gleicher Schlüssel aufbrechen
\end{itemize}

\subsection{Sondierung}

\begin{description}[labelindent=16pt,style=multiline,leftmargin=5cm, noitemsep]
	\item[primary clustering:] lang belegte Teilstücke haben also eine grössere Tendenz zu wachsen (linear)
	\item[secondary clustering:] Synonyme durchlaufen die selbe Sondierungsfolge (linear und quadratisch)
\end{description}

\subsubsection{Linear}

\begin{equation*}
	h(k), h(k)-1, h(k)-2, ..., 0, m-1, ..., h(k)+1
\end{equation*}

\subsubsection{Quadratisch}

\begin{equation*}
	h(k), h(k)+1, h(k)-1, h(k)+4, h(k)-4, h(k)+9, h(k)-9
\end{equation*}

\subsubsection{Double Hashing}

\begin{equation*}
	h(k), h(k)-h'(k), h(k)-2h'(k), ..., h(k)-(m-1)h'(k)
\end{equation*}

Genau so effizient wie uniformes Sondieren.

\section{Datenstrukturen}

\subsection{Union-Find Struktur}

\paragraph{Vereinigung nach Grösse/Höhe:} Man macht die Wurzel des Baumes mit kleinerer Größe (bzw. geringerer Höhe) zum direkten weiteren Sohn des Baumes mit der größeren Größe (bzw. Höhe). \\
Die Grösse/Höhe eines Teilbaums wird in der Wurzel gespeichert.

\paragraph{Pfadverkürzung:} Bei der Find-Operation werden die durchlaufenen Knoten direkt an die Wurzel gehängt. Die Operation wird doppelt so teuer.

\subsection{Suchbäume}

\begin{description}[labelindent=16pt,style=multiline,leftmargin=3cm, noitemsep]
	\item[preorder:] Hauptreihenfolge \\
					 Wurzel, linker Teilbaum, rechter Teilbaum	\\
					 Rekonstruktion: In selber Reihenfolge in leeren Suchbaum einfügen
	\item[postorder:] Nebenreihenfolge \\
					  linker Teilbaum, rechter Teilbaum, Wurzel \\
					  Rekonstruktion: In umgekehrter Reihenfolge in leeren Suchbaum einfügen
	\item[inorder:] symmetrische Reihenfolge \\
					linker Teilbaum, Wurzel, rechter Teilbaum
\end{description}

\subsubsection{AVL-Baum}

Für jeden Knoten p des Baumes gilt, dass sich die Höhe des linken Teilbaumes von der Höhe des rechten Teilbaumes von p höchstens um 1 unterscheidet. \\

Das Einfügen eines neuen Elements führt höchstens zu einer Rotation.

\subsubsection{Splay-Baum}

Suchbaum, der die Move-to-Root Operation unterstützt. Ein Knoten wird zur Wurzel hochgeschoben, in dem Rotationen wie beim AVL-Baum durchgeführt werden.

\subsubsection{B-Baum}

Für ein B-Baum der Ordnung $m$ gilt:

\begin{itemize}[noitemsep]
	\item Alle Blätter haben die gleiche Tiefe
	\item Jeder Knoten mit Ausnahme der Wurzel und der Blätter hat wenigstens $\lceil \frac{m}{2} \rceil$ Söhne
	\item Die Wurzel hat wenigstens 2 Söhne
	\item Jeder Knoten hat höchstens $m$ Söhne
	\item Jeder Knoten mit $i$ Söhnen hat $i-1$ Schlüssel
	\item Knoten dürfen höchstens $m-1$ Schlüssel speichern
\end{itemize}

\subsubsection{Segment-Baum}

\paragraph{Aufspiessanfrage:} Es werden alle Intervalle ausgegeben, die auf dem Suchpfad zum Elementarsegment gehören

\subsubsection{Heap}

\paragraph{Versickern:} Max/Min mit letztem Element austauschen, Blatt entfernen, Wurzel so lang mit Max der Kinder austauschen, bis es grösser als diese ist (oder Blatt)

\section{Sortieralgorithmen}

\begin{table}[H]
\centering
\begin{tabular}{|l|l|l|l|l|l|}
\hline
\textbf{Name}                            & \textbf{Best Case} & \textbf{Average Case} & \textbf{Wort Case} & \textbf{in-situ} & \textbf{stabil} \\ \hline
Selectionsort                            & $O(n^2)$           & $O(n^2)$              & $O(n^2)$           & ja                & nein            \\ \hline
Bubblesort / Insertionsort & $O(n)$             & $O(n^2)$              & $O(n^2)$           & ja                & ja              \\ \hline
Heapsort                                 & $O(n \log n)$      & $O(n \log n)$         & $O(n^2)$           & ja                & nein            \\ \hline
Mergesort                                & $O(n \log n)$      & $O(n \log n)$         & $O(n \log n)$      & $O(n)$            & ja              \\ \hline
Quicksort                                & $O(n \log n)$      & $O(n \log n)$         & $O(n^2)$           & $O(\log n)$       & nein     
	\\ \hline      
\end{tabular}
\end{table}

\subsection{Selectionsort}

\begin{enumerate}[noitemsep]
	\item Suche Minimum der letzten $n$ Elemente
	\item Tausche Minimum mit dem ersten Element
	\item Fahre fort mit den $n-1$ letzten Elementen
\end{enumerate}

\subsection{Insertionsort}

\begin{enumerate}[noitemsep]
	\item Nächstes Element betrachten
	\item Position für betrachtetes Element in sortierter Teilfolge (prefix) finden und einfügen
\end{enumerate}

\subsection{Quicksort}

\begin{enumerate}[noitemsep]
	\item Pivot auswählen
	\item Teilfolgen erstellen, von links und rechts in die Mitte iterieren, Elemente mit einander austauschen falls nötig
	\item Pivot mit mittlerem Element tauschen
\end{enumerate}

\subsection{Mergesort}

\begin{enumerate}[noitemsep]
	\item Array halbieren
	\item Mergesort rekursiv auf beide Teilfolgen aufrufen
	\item Sortierte Teilfolgen verschmelzen	
\end{enumerate}

\section{Graphenalgorithmen}

\subsection{Tiefen- und Breitensuche}

\paragraph{Laufzeit:} $O(|E|+|V|)$

\subsection{Matching}

\begin{description}[labelindent=16pt,style=multiline,leftmargin=5cm, noitemsep]
	\item[Zuordnung] Teilmenge an Kanten, sodass keine zwei Kanten denselben Endknoten haben
	\item[Grösse] Anzahl Kanten in der Zuordnung, also $|Z|$
\end{description}

\subsection{Minimaler Spannbaum}

\subsubsection{Kruskal}

\paragraph{Laufzeit:} $O(|E|\log|V|)$

Die billigste Kante wird betrachtet, verbindet sie zwei Teilmengen von Knoten, so wird sie hinzugefügt (Union-Find-Struktur)

\subsubsection{Dijkstra}

\paragraph{Laufzeit:} $O(|E|+|V|\log|V|)$

Tiefensuche, es wird immer die billigste Kante ausgewählt

\subsection{Topologische Sortierung}

\begin{center}
	Für $G$ gibt es eine topologische Sortierung $\Leftrightarrow$ $G$ ist zyklenfrei
\end{center}

\paragraph{Laufzeit:} $O(|E|+|V|)$

\begin{enumerate}[noitemsep]
	\item Wähle ein Knoten aus
	\item Besuch die Kinder, passe deren Ordnung an
	\item Wähle nächsten nicht besuchten Knoten
\end{enumerate}

\subsection{Kürzeste Wege}

\subsubsection{One-to-One (Dijkstra)}

\paragraph{Bemerkung:} Es sind nur positive Kanten erlaubt. Für die Priority Queue kann eine Fibonacci-Heap gebraucht werden.

\paragraph{Laufzeit:} $|E| + |V| \log|V|$

\begin{enumerate}[noitemsep]
	\item Wähle Anfangsknoten aus
	\item Wähle nächsten Knoten aus, der minimale Distanz zu Anfangsknoten hat
	\item Passe neue Distanz aller Randknoten an
	\item Wähle nächsten Knoten
\end{enumerate}

\subsubsection{One-to-All (Bellman/Ford)}

\paragraph{Bermerkung:} Funktioniert auch bei Digraphen mit negativen Pfaden.

\paragraph{Laufzeit:} $O(|E|\cdot|V|)$ \\

Funktioniert ähnlich wie nach Dijkstra. Knoten werden jedoch nicht endgültig gewählt, sondern mehrmals besucht. Beendet ist der Algorithmus, wenn sich kein Knoten mehr updaten lässt.

\subsubsection{All-to-All}

\paragraph{Laufzeit:} $O(|V|\cdot (|E| + |V|\log|V|))$

\begin{enumerate}[noitemsep]
	\item Transformiere allgemeinen Graphen (mit möglichen negativen Kanten) zu Distanzgraphen
	\item Wende Bellman/Ford auf Graphen an
\end{enumerate}

\subsection{Flussnetzwerke}

\subsection{Ford/Fulkerson}

\paragraph{Laufzeit:} $O(|E|f)$, mit dem minimalen Schnitt $f$ 

\subsection{Edmonds/Karp}

Angepasster Ford/Fulkerson Algorithmus.

\paragraph{Laufzeit:} $O(|V|\cdot|E|^2)$

\begin{enumerate}[noitemsep]
	\item Suche kürzeste Weg von Quelle zur Senke
	\item Benutze diesen Pfad mit maximal möglicher Kapazität
	\item Wiederhole so lang, bis es keinen Pfad mer zur Senke gibt
\end{enumerate}

\subsection{Dinic}

Ähnlich wie Edmonds/Karp

\paragraph{Laufzeit:} $O(|V|^2\cdot|E|)$

\section{Geometrische Algorithmen}

\subsection{Scanline}

\begin{description}
	\item[Sichtbarkeit:] AVL-Tree
\end{description}

\end{document}

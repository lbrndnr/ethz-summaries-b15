\documentclass[11pt]{article}

\usepackage{amsmath}
\usepackage{amssymb}
\usepackage{array}
\usepackage{geometry}
\usepackage{enumitem}
\usepackage{float}
\usepackage{cancel}
\usepackage{graphicx}
\usepackage[labelformat=empty]{caption}
\usepackage{booktabs}

\geometry{
	a4paper,
 	left=20mm,
 	top=20mm,
}

\setlength{\parindent}{0pt}

\begin{document}

\section{Allgemein}

\subsection{Trigonometrie}

\begin{center}
	\includegraphics[width=\linewidth,keepaspectratio=true]{images/trigonometry}
\end{center}

\begin{center}
	\includegraphics[width=300pt]{images/trigonometry_table}
\end{center}

\begin{equation*}
\begin{split}
	\tan(x) & = \frac{\sin(x)}{\cos(x)} \\
	\sin(2x) & = 2\sin(x)\cos(x) \\
	\sin(\alpha + \beta) & = \sin(\alpha)\cos(\beta) + \cos(\alpha)\sin(\beta) \\
	\cos(\alpha + \beta) & = \cos(\alpha)\cos(\beta) + \sin(\alpha)\sin(\beta) \\
	\cos^2(x) & = \frac{1 + \cos(2x)}{2} \\
	\sin^2(x) + \cos^2(x) & = 1
\end{split}
\end{equation*}

\subsection{Potenzgesetze}

\begin{equation*}
\begin{split}
	a^0 & = 1 \\
	a^1 & = a \\
	a^m \cdot a^n & = a^{m+n} \\
	(a^n)^m & = a^{nm} \\
	\frac{a^n}{a^m} & = a^{n-m} \\
	a^{-n} & = \frac{1}{a^n} \\
	a^{\frac{b}{n}} & = \sqrt[n]	{a^b} \\
\end{split}
\end{equation*}

\section{Mengen}

\subsection{Definitionen}

\begin{description}[labelindent=16pt,style=multiline,leftmargin=6cm, noitemsep]
	\item[Obere/Untere Schranke:] $\exists b \in \mathbb{R}\ \forall a\in A:\ a \leq b$, $\exists c \in \mathbb{R}\ \forall a\in A:\ a \geq c$
	\item[Supremum:] kleinste obere Schranke $\sup A$
	\item[Infimum:] gr{\"o}sste untere Schranke $\inf A$
	\item[Maximum/Minimum:] $\sup A \in A$, $\inf A \in A$
\end{description}

\subsubsection{Vorgehen zur Bestimmung von Maximum/Minimum}

\begin{enumerate}[noitemsep]
	\item Zeigen, dass $f(x)$ stetig ist
	\item Zeigen, dass Definitionsmenge kompakt ist
	\item Nach \textbf{Satz von Weierstrass} wird Maximum/Minimum angenommen
	\item Maximum/Minimum bestimmen
\end{enumerate}

\subsection{Identit{\"a}ten}

\begin{equation*}
\begin{split}
	A + B & := \{a + b | a \in A, b \in B\} \\
	\sup(A+B) = \sup A + \sup B,\ & \inf(A+B) = \inf A + \inf B \\
	\sup(A \cup B) = \max\{\sup A, \sup B\},\ & \inf(A \cup B) = \min\{\inf A, \inf B\}
\end{split}
\end{equation*}

\section{Komplexe Zahlen}

\subsection{Polarform}

\begin{minipage}[c]{0.5\textwidth}
\centering
\includegraphics[width=\linewidth,keepaspectratio=true]{images/polarform}
\end{minipage}
%
\begin{minipage}[c]{0.5\textwidth}
\begin{equation*}
\begin{split}
	z & = x + iy = r(\cos(\varphi) + i\sin(\varphi)) = re^{i\varphi} \\
	r & = |z| = \sqrt{x^2 + y^2} \\
	\varphi & = \arctan(\frac{y}{x}) \quad \text{(je nach Quadrant)}  \\
	x & = r\cos(\varphi) \\
	y & = r\sin(\varphi) \\
	zw & = (re^{i\varphi})\cdot(se^{i\psi}) = rse^{i(\varphi + \psi)} \\
	\sqrt[q]{z} & = \sqrt[q]{s}e^{i\phi}\text{, wobei }\phi = \frac{\varphi}{q} \mod \frac{2\pi}{q} \\
	e^{i(\frac{\pi}{2} + 2\pi k)} & = i,\ e^{i\pi} = 1, \ e^{-i\pi} = -1
\end{split}
\end{equation*}
\end{minipage}

\subsection{Identit{\"a}ten}

\begin{equation*}
\begin{split}
	\overline{z} & = x - iy\\
	z^{-1} & = \frac{\overline{z}}{|z|^2} \\
	(a,b) \cdot (c, d) & = (ac-bd, ad+bc) \\
	i & = \sqrt{-1}\\
	i^2 & = -1 \\
	|z|^2 & = z\overline{z} \\
	|zw|^2 & = (zw) \cdot \overline{(zw)} = |z|^2|w|^2
\end{split}
\end{equation*}

\section{Grenzwert}

\subsection{Dominanz}

\begin{equation*}
\begin{split}
	\text{F{\"u}r}\ x \to +\infty:\quad & ... < \log(\log(x)) < \log(x) < x^\alpha < \alpha^x < x! < x^x \\
	\text{F{\"u}r}\ x \to 0:\quad & ... < \log(\log(x)) < \log(x) < (\frac{1}{x})^\alpha \\
\end{split}
\end{equation*}

\subsection{Fundamentallimes}

\begin{equation*}
\begin{split}
	\lim_{x \to a} \frac{\sin \odot}{\odot} = \lim_{x \to a} \frac{\tan \odot}{\odot} & = 1\ \text{mit}\ \odot \xrightarrow{\: x \to a \: } 0 \\ 
	\lim_{x \to a} (1 + \frac{1}{\odot})^\odot & = e\ \text{mit}\ \odot \xrightarrow{\: x \to a \: } \infty \\ 
	\lim_{x \to a} (1 + \odot)^\frac{1}{\odot} & = e\ \text{mit}\ \odot \xrightarrow{\: x \to a \: } 0 \\ 
\end{split}
\end{equation*}

\subsection{Wurzeltrick}

\begin{equation*}
	\lim_{x\to\infty} \sqrt{\alpha}+\beta = \lim_{x\to\infty}(\sqrt{\alpha}+\beta)\frac{\sqrt{\alpha}-\beta}{\sqrt{\alpha}-\beta}
\end{equation*}

\subsection{$e^{\log(x)}$-Trick}

\paragraph{Anforderung:}Term der Form $f(x)^{g(x)}$ mit Grenzwert "$0^0$", "$\infty^0$" oder "$1^\infty$" f{\"u}r $x \to 0$

\begin{equation*}
	\textbf{Grundsatz:}\quad\lim_{x\to a}f(x)^{g(x)} = \lim_{x\to a}e^{g(x) \cdot \log(f(x))}
\end{equation*}

\emph{Tipp:} Danach den Limes des Exponenten berechnen. Oft ist Bernoulli-de l'H{\^o}pital dazu n{\"u}tzlich.

\subsection{Satz von Bernoulli-de l'H{\^o}pital}

\paragraph{Anforderung:}Term der Form $\frac{f(x)}{g(x)}$ mit Grenzwert entweder "$\frac{0}{0}$" oder "$\frac{\infty}{\infty}$" mit $g'(x) \neq 0$. \\

\begin{equation*}
	\textbf{Grundsatz:}\quad\lim_{x\to a}\frac{f(x)}{g(x)} = \lim_{x\to a}\frac{f'(x)}{g'(x)}
\end{equation*}

\begin{table}[H]
\centering
\begin{tabular}{|c|c|c|}
\hline
\textbf{Term} & \textbf{Anforderung} & \textbf{Umformung} \\ \hline
$f(x)g(x)$              & "$0\cdot\infty$"                     & $\frac{g(x)}{\frac{1}{f(x)}}$          \\ \hline 
$\frac{f(x)}{g(x)} - \frac{h(x)}{i(x)}$ & "$\infty - \infty$"  & $\frac{f(x)i(x) - h(x)g(x)}{g(x)i(x)}$ \\ \hline      
\end{tabular}
\end{table}

\subsection{Wichtige Grenzwerte}

\begin{equation*}
\begin{split}
	\lim\limits_{n \to \infty} \left( 1+\frac{x}{n} \right)^n = e^x \qquad & \qquad \lim\limits_{n \to \infty} \left( 1+\frac{1}{n} \right)^n = e \\
	\lim\limits_{x \to 0} \frac{a^x-1}{x} = \ln a \qquad & \qquad \lim\limits_{x \to 0} \frac{\log_a(1+x)}{x} = \frac{1}{\ln a} \\
	\lim\limits_{x \to 0} \frac{1-\cos(x)}{x} = 0 \quad \qquad & \qquad \lim\limits_{x \to 0} \frac{1-\cos(x)}{x^2} = \frac{1}{2} \\
	\lim\limits_{x \to 0} \frac{\tan(x)}{x} = 1 \qquad & \qquad \lim\limits_{x \to 0} \frac{\sin(x)}{x} = 1 \\
	\lim\limits_{n \to \infty} \frac{n!}{n^n} = 0 \qquad & \qquad \lim\limits_{n \to 0} \frac{e^n -1 }{n} = 1 \\
	\lim\limits_{n \to \infty} \sqrt[n]{n!} = \infty \qquad & \qquad \lim\limits_{n \to \infty} \sqrt[n]{n} = 1 \\
	\lim\limits_{n \to \infty} \ln(n) = \infty \qquad & \qquad \lim\limits_{x \to 0} \frac{\log_a(1+x)}{x} = \frac{1}{\ln a} \\
\end{split}
\end{equation*}

\section{Folgen}

\subsection{Definition}

\begin{equation*}
\begin{split}
	\textbf{Konvergenz:} \quad & \forall \varepsilon > 0\ \exists N = N(\varepsilon) \in \mathbb{N},\ \text{sodass}\ \forall n \geq N: |a_n - a| < \varepsilon \\
	\textbf{Divergenz:} \quad & \forall K > 0\ \exists N = N(K) \in \mathbb{N},\ \text{sodass}\ \forall n \geq N: |a_n| > K
\end{split}
\end{equation*}

\subsection{Beweis}

Entweder \textbf{Grenzwert berechnen}, oder:

\begin{enumerate}[noitemsep]
	\item Zeige mittels \textbf{Induktion}, dass die Folge \textbf{beschr{\"a}nkt} ist und monoton \textbf{steigt/f{\"a}llt}. Benutze dazu z.B. folgende Aussagen: $a_n \leq a_{n+1}$ oder $a_{n+1}-a_n \geq 0$.
	\item Sch{\"a}tze den Grenzwert durch die ersten paar Terme ab
	\item Beweise den Grenzwert (z.B. mit $a_n \geq a$)
\end{enumerate}

\section{Reihen $\sum^\infty$}

\subsection{Konvergenzkriterien}

\begin{table}[H]
\centering
\begin{tabular}{|p{5cm}|p{4cm}|p{5cm}|}
\hline
                                             & \textbf{Eignung}    & \textbf{Bemerkung}                        \\ \hline
\textbf{Limes des allgemeinen Glieds}        &                     & zeigt nur Divergenz                       \\ \hline
\textbf{Majoranten- und Minorantenkriterium} &                     & ersten Glieder spielen keine Rolle        \\ \hline
\textbf{Quotientenkriterium}                 & $a_n$ mit Faktoren wie $n!$, $a^n$, oder Polynome & gleiche Folgerung wie Wurzelkriterium     \\ \hline
\textbf{Wurzelkriterium}                     & $a_n = (b_n)^n$     & gleiche Folgerung wie Quotientenkriterium \\ \hline
\textbf{Leibnitz-Kriterium}                  & $\sin$, $\cos$, $\tan$, $(-1)^n$ &                                           \\ \hline
\textbf{Absolute Konvergenz}                 & $\sin$, $\cos$, $\tan$, $(-1)^n$   &                                           \\ \hline
\textbf{Sandwich-Theorem}					 & $\sin$, $\cos$, $\tan$, $(-1)^n$ & \\ \hline
\end{tabular}
\end{table}
\subsubsection*{Limes des allgemeinen Glieds}

\paragraph{Bemerkung:} Mit dieser Methode l{\"a}sst sich nur die Divergenz beweisen, nicht jedoch die Konvergenz.

\begin{enumerate}
	\item $\sum_n a_n$ gegeben
	\item Grenzwert $\lim_{n\mapsto\infty} a_n$ berechnen
	\begin{itemize}
		\item falls Grenzwert $\neq 0 \Rightarrow$ \textbf{divergent} 
		\item falls Grenzwert $= 0 \Rightarrow$ keine Aussage 
	\end{itemize}
\end{enumerate}

\subsubsection*{Majoranten- und Minorantenkriterium}

Es seien $a_n$, $b_n > 0$ mit $a_n \geq b_n\ \forall n$ ab einem gewissen $n_0$. Dann gilt: 
\begin{equation*}
\begin{split}
	\sum_n a_n \text{ konvergent} & \Rightarrow \sum_n b_n \textbf{ konvergent}\quad \text{(Majorantenkriterium)} \\
	\sum_n b_n \text{ divergent} & \Rightarrow \sum_n a_n \textbf{ divergent}\quad \text{(Minorantenkriterium)} \\
\end{split}
\end{equation*}

\subsubsection*{Vergleichskriterium}

\begin{enumerate}
	\item $\sum_n a_n$ und $\sum_n b_n$ gegeben mit $a_n,b_n > 0$
	\item Grenzwert $\lim_{n\mapsto\infty} \frac{a_n}{b_n}$ berechnen
	\begin{itemize}
		\item falls Grenzwert $= 0$:
		\begin{itemize}
			\item $\sum_n a_n$ divergent $\Rightarrow \sum_n b_n$ \textbf{divergent}
			\item $\sum_n b_n$ konvergent $\Rightarrow \sum_n a_n$ \textbf{konvergent}
		\end{itemize} 
		\item falls Grenzwert $= \infty$:
		\begin{itemize}
			\item $\sum_n a_n$ konvergent $\Rightarrow \sum_n b_n$ \textbf{konvergent}
			\item $\sum_n b_n$ divergent $\Rightarrow \sum_n a_n$ \textbf{divergent}
		\end{itemize} 
	\end{itemize}
\end{enumerate}

\subsubsection*{Quotientenkriterium}

\begin{enumerate}
	\item $\sum_n a_n$ mit $a_n \neq 0$ gegeben
	\item Grenzwert $\lim_{n\mapsto\infty}|\frac{a_{n+1}}{a_n}|$ berechnen
	\begin{itemize}
		\item falls Grenzwert $> 1 \Rightarrow$ \textbf{divergent}
		\item falls Grenzwert $< 1 \Rightarrow$ \textbf{konvergent}
		\item falls Grenzwert $= 1 \Rightarrow$ keine Aussage
	\end{itemize}
\end{enumerate}

\subsubsection*{Wurzelkriterium}

\begin{enumerate}
	\item $\sum_n a_n$ mit $a_n \neq 0$ gegeben
	\item Grenzwert $\lim_{n\mapsto\infty}\sqrt[n]{|a_n|}$ berechnen
	\begin{itemize}
		\item falls Grenzwert $> 1 \Rightarrow$ \textbf{divergent}
		\item falls Grenzwert $< 1 \Rightarrow$ \textbf{konvergent}
		\item falls Grenzwert $= 1 \Rightarrow$ keine Aussage
	\end{itemize}
\end{enumerate}

\subsubsection*{Leibniz-Kriterium}

\begin{enumerate}
	\item $\sum_n (-1)^n a_n$ gegeben
	\item \textbf{konvergent}, falls:
	\begin{enumerate}
		\item $a_n \geq 0$
		\item $\lim_{n\mapsto\infty} a_n = 0$
		\item $a_n$ monoton fallend
	\end{enumerate}
\end{enumerate}

\subsubsection*{Absolute Konvergenz}

\begin{enumerate}
	\item $\sum_n (-1)^n a_n$ gegeben
	\item \textbf{konvergent}, falls $\sum_n |a_n|$ konvergent
\end{enumerate}

\subsection{Geometrische Reihe}

Reihe der Form $\sum^\infty_{k = 0} a \cdot r^k$ mit der \textbf{Partialsumme}:

\begin{equation*}
	S_N=\frac{a-ar^{N+1}}{1-r}
\end{equation*}

\textbf{Konvergent}, falls $0<|r|<1$ mit Grenzwert:

\begin{equation*}
	\sum^\infty_{k=0}ar^k=\frac{a}{1-r}
\end{equation*}

\subsection{Potenzreihe}

Reihe der Form $\sum^\infty_0 a_nx^n$. \textbf{Konvergent}, falls $|x|<\rho$. In diesem Gebiet darf man die Reihe ableiten und integrieren.

\begin{equation*}
	\rho= \lim_{n\rightarrow \infty}|\frac{a_n}{a_{n+1}}|
\end{equation*}
\begin{equation*}
	\rho=\frac{1}{\lim_{n\rightarrow \infty}\sqrt[n]{|a_n|}}
\end{equation*}

\paragraph{Konvergenzverhalten am Rand:} Es muss noch überprüft werden, ob die Reihe für genau $\rho$ konvergiert. Dazu muss $\rho$ in die Formel eingesetzt werden.

\subsubsection{Tipps}
\begin{equation*}
\begin{split}
	\cos(x) & = \sum^\infty_{n =0}\frac{(-1)^nx^{2n}}{(2n)!} \\
	\sin(x) & = \sum^\infty_{n =0}\frac{(-1)^nx^{2n+1}}{(2n+1)!} \\
	e^x & = \sum^\infty_{n =0}\frac{x^n}{n!} \\
	n^2 & = \sum_{k=1}^n 2k-1\\
	harmonische: \sum_{n=1}^{\infty}\frac{1}{n}=\infty
\end{split}
\end{equation*}

\subsubsection{Potenzreihenentwicklung}

\begin{equation*}
	\textbf{Grundsatz:} \quad f(x) = \sum_{n=0}^\infty \frac{f^{(n)}(0)}{n!} \cdot x^n
\end{equation*}

\section{Stetigkeit}

\subsection{Stetigkeitskriterien}

\subsubsection*{Weierstrass-Kriterium}
F{\"u}r alle $\epsilon > 0$ gibt es ein $\delta(\epsilon, a) >0$, sodass f{\"u}r alle $|x-a|<\delta$ gilt:
\begin{equation*}
	|f(x) -f(a)|<\epsilon
\end{equation*}

\subsubsection*{Gleichm{\"a}ssige Stetigkeit}
F{\"u}r alle $\epsilon > 0$ gibt es ein $\delta(\epsilon) >0$, sodass f{\"u}r alle $|x-y|<\delta$ gilt:
\begin{equation*}
	|f(x)-f(y)| < \epsilon
\end{equation*}
\emph{Bemerkung:} Ist $f$ \textbf{stetig und kompakt}, dann ist sie auch gleichm{\"a}ssig stetig.

\subsubsection*{Lipschitz-Stetigkeit}
Es existiert eine Konstante $L\in \mathbb{R}$, sodass:
\begin{equation*}
	|f(x)-f(y)|\leq L|x-y| \quad \forall x,y \in \Omega
\end{equation*}

\emph{Bemerkung:} Ist $f'$ \textbf{auf $\Omega$ beschr{\"a}nkt}, so ist $f$ Lipschitz-stetig. Lipschitz-Stetigkeit impliziert gleichm{\"a}ssige Stetigkeit.

\subsubsection*{Punktweise Konvergenz}
$f_n(x)$ konvergiert punktweise falls:
\begin{equation*}
	\forall x\in \Omega \quad \lim_{n\rightarrow\infty}f_n(x) = f(x)
\end{equation*}

\subsubsection*{Gleichm{\"a}ssige Konvergenz}

\paragraph{Grundsatz:} Falls eine Folge stetiger Funktionen $f_n$ gleichm{\"a}ssig gegen $f$ konvergiert, ist $f$ stetig.\\

$f_n(x)$ konvergiert gleichm{\"a}ssig falls:
\begin{equation*}
	\lim_{n\rightarrow\infty} \sup|f_n(x) - f(x)| = 0
\end{equation*}

\emph{Bemerkung:} Gleichm{\"a}ssige Konvergenz impliziert punktweise Konvergenz.

\section{Differenzialrechnung}

Eine stetige Funktion ist differenzierbar, falls der Grenzwert $f'(x_0)$ existiert:

\begin{equation*}
	f'(x_0) := \lim_{x\to x_0}\frac{f(x) - f(x_0)}{x-x_0}
\end{equation*}

\subsection{Umkehrsatz}

\begin{equation*}
	(f^{-1})'(y) = \frac{1}{f'(x)}
\end{equation*}

\subsection{Mittelwertsatz}

\begin{equation*}
	f'(c) = \frac{f(b) - f(a)}{b - a}
\end{equation*}

\subsection{Taylorpolynom}

Das Taylorpolynom $m$-ter Ordnung von $f(x)$ an der Stelle $x=a$
\begin{equation*}
	P^a_m(x) := f(a) + f'(a)(x-a) + \frac{1}{2}f''(a)(x-a)^2 + ... + \frac{1}{m!} f^{(m)}(a)(x-a)^m
\end{equation*}

mit dem Fehlerterm $R^a_m(x)$, wobei $\xi$ zwischen $a$ und $b$ liegt:
\begin{equation*}
	R^a_m(x) = \frac{f^{(m+1)}(\xi)}{(m+1)!}(x+a)^{m+1},\ \text{wobei}\ f(x) = P^a_m(x) + R^a_m(x)
\end{equation*}

\subsection{Hauptsatz von Calculus}
\begin{equation*}
	f(x)=\int^{m(x)}_lg(t)dt
\end{equation*}
\begin{equation*}
	f'(x)=g(m(x))\cdot\frac{d}{dx}m(x)
\end{equation*}
wobei $m(x)$ der Form $ax^b$ ist mit $l\in \mathbb{R}$

\section{Integration}

\subsection{Elementare Integrale}

\begin{table}[H]
\centering
\begin{tabular}{|c|c|c|}
\hline
$f'(x)$ & $f(x)$ & $F(x)$ \\ \specialrule{.1em}{0em}{0em} 
$0$ & $c$ & $cx$ \\ \hline
$r\cdot x^{r-1}$ & $x^r$ & $\frac{x^{r+1}}{r+1}$ \\ \hline
$-\frac{1}{x^2} = -x^{-2}$ & $\frac{1}{x} = x^{-1}$ & $\ln|x|$ \\ \hline
$\frac{1}{2\sqrt{x}} = \frac{1}{2}x^{-\frac{1}{2}}$ & $\sqrt{x} = x^{\frac{1}{2}}$ & $\frac{2}{3}x^\frac{3}{2}$ \\ \hline
$\cos(x)$ & $\sin(x)$ & $-\cos(x)$ \\ \hline
$-\sin(x)$ & $\cos(x)$ & $\sin(x)$ \\ \hline
$1 + \tan^2(x) = \frac{1}{\cos^2(x)}$ & $\tan(x)$ & $-\ln|\cos(x)|$ \\ \hline
$e^x$ & $e^x$ & $e^x$ \\ \hline
$c\cdot e^{cx}$ & $e^{cx}$ & $\frac{1}{c}\cdot e^{cx}$ \\ \hline
$\ln(c)\cdot c^x$ & $c^x$ & $\frac{c^x}{\ln(c)}$ \\ \hline
$\frac{1}{x}$ & $\ln|x|$ & $x(\ln|x| - 1)$ \\ \hline
$\frac{1}{\ln(a) \cdot x}$ & $\log_a|x|$ & $\frac{x}{\ln(a)}(\ln|x| -1)$ \\ \hline
$\frac{1}{\sqrt{1-x^2}}$ & $\arcsin(x)$ & $x\cdot\arcsin(x) + \sqrt{1-x^2}$ \\ \hline
$-\frac{1}{\sqrt{1-x^2}}$ & $\arccos(x)$ & $x\cdot\arccos(x) - \sqrt{1-x^2}$ \\ \hline
$\frac{1}{1+x^2}$ & $\arctan(x)$ & $x\cdot \arctan(x) - \frac{1}{2}\ln(1+x^2)$ \\ \hline
$\cosh(x)$ & $\sinh(x) = \frac{e^x - e^{-x}}{2}$ & $\cosh(x)$ \\ \hline
$\sinh(x)$ & $\cosh(x) = \frac{e^x + e^{-x}}{2}$ & $\sinh(x)$ \\ \hline
$\frac{1}{\cosh^2(x)}$ & $\tanh(x)$ & $\log(\cosh(x))$ \\ \hline
\end{tabular}
\end{table}

\subsection{Regeln}

\begin{equation*}
\begin{split}
	\textbf{Direkter Integral}\quad & \int f(g(x))g'(x)\ dx = F(g(x)) \\
	\textbf{Partielle Integration}\quad & \int f' \cdot g\ dx = f \cdot g - \int f \cdot g'\ dx \\
	\textbf{mit Polynomen}\quad & \int\frac{p(x)}{q(x)}\ dx \Rightarrow\ \text{Partialbruchzerlegung} \\
	\textbf{Substitution}\quad & \int_a^b f(\varphi(t))\varphi'(t)\ dt = \int_{\varphi(a)}^{\varphi(b)} f(x)\ dx\ \text{mit}\ x = \varphi(t)
\end{split}
\end{equation*}

\subsection{Tipps}

\begin{equation*}
\begin{split}
	\int\tan x\ dx & = \int\frac{\sin x}{\cos x}\ dx = -\log|\cos(x)| \\
	\int \frac{1}{x - \alpha}\ dx & = \log(x-\alpha) \\
	\int\frac{\frac{1}{\alpha}}{1+(\frac{x}{\alpha})^2}\ dx & = \arctan(x) \\
	\int \sin^2(x)\ dx & = \frac{1}{2}(x - \sin(x)\cos(x)) + C \\
	\int \cos^2(x)\ dx & = \frac{1}{2}(x + \sin(x)\cos(x)) + C \\
	\int \sqrt{x^2+1}\ dx & = \sinh(x) + C
\end{split}
\end{equation*}

\subsection{Uneigentliche Integrale}

\begin{equation*}
\begin{split}
	\int_0^\infty f(x)\ dx = \lim_{R \to \infty} \int_0^R f(x)\ dx \\
	\int_{-\infty}^\infty f(x)\ dx = \lim_{R \to -\infty} \int_R^k f(x)\ dx + \lim_{R \to \infty} \int_k^R f(x)\ dx
\end{split}
\end{equation*}

Gibt es eine Unstetigkeitstelle $c$ in dem Integrationsgebiet, so geht man wie folgt vor:
\begin{equation*}
	\int_a^b f(x)\ dx = \lim_{\varepsilon \to 0} \int_a^{c-\varepsilon} f(x)\ dx + \lim_{\varepsilon \to 0} \int_{c+\varepsilon}^b f(x)\ dx
\end{equation*}

\section{Differentialgleichungen}

\subsection{Grundbegriffe}

\begin{description}[labelindent=16pt,style=multiline,leftmargin=3.5cm, noitemsep]
	\item[Ordnung:] h{\"o}chste vorkommende Ableitung
	\item[linear:] alle $y$-abh{\"a}ngigen Terme kommen linear vor (keine Terme wie zum Beispiel $y^2$, $(y'')^3$, $\sin(y)$, $e^{y'}$)
	\item[homogen:] Gleichung ohne St{\"o}rfunktionen
	\item[St{\"o}rfunktion:] Term, der rein von der Funktionsvariablen $x$ abh{\"a}ngt
\end{description}

\subsection{Methoden}

\begin{table}[H]
\centering
\begin{tabular}{|p{5cm}|p{6cm}|p{3cm}|}
\hline
                                  	& \textbf{Problem} 							& \textbf{Anforderungen} 			\\ \hline
\textbf{Trennung der Variablen}   	& $y' = \frac{dy}{dx} = h(x) \cdot g(y)$ 	& 1. Ordnung			            \\ \hline
\textbf{Variation der Konstanten}	& $y' = \frac{dy}{dx} = h(x)y + b(x)$	 	& 1. Ordnung \filbreak inhomogen	\\ \hline
\textbf{Euler-Ansatz}				& $a_{n}y^{(n)} + a_{n-1}y^{(n-1)} + ... + a_{0}y = 0$	 	& n. Ordnung \filbreak linear \filbreak homogen	\\ \hline
\textbf{Direkter Ansatz}				& $a_{n}y^{(n)} + a_{n-1}y^{(n-1)} + ... + a_{0}y = b(x)$	& n. Ordnung \filbreak linear \filbreak inhomogen	\\ \hline

%\textbf{Substitution}				& $y' = h(\frac{y}{x})$ \filbreak $y' = h(ax + by + c)$ \filbreak $y' = h(\frac{ax + by + c}{dx + ey + f})$ \filbreak $y' = \frac{y}{x}h(xy)$ 																		& nicht direkt separierbar			\\ \hline
\end{tabular}
\end{table}

\subsubsection{Trennung der Variable}

\begin{equation*}
\begin{split}
	& y' + x \tan y = 0,\ y(0) = \frac{\pi}{2} \\
	\text{umformen}\quad & \frac{dy}{dx} = -x \tan y \\
	\textbf{konstante L{\"o}sungen}\quad & y(x) \equiv 0\ \text{erf{\"u}llt jedoch $y(0) \equiv \frac{\pi}{2}$ nicht} \\
	\text{Trennung}\quad & \frac{dy}{\tan y} = -x dx \\
	\text{integrieren}\quad & \int\frac{\cos y}{\sin y}dy = - \int xdx \Rightarrow \log|\sin y| = -\frac{x^2}{2} + C \\
	& \Rightarrow |\sin y| = e^Ce^{\frac{-x^2}{2}} \Rightarrow \sin y = \pm e^Ce^{\frac{-x^2}{2}} = Ce^{\frac{-x^2}{2}} \\
	\text{Anfangsbedingung gebrauchen}\quad & \sin(y(0)) = \sin (\frac{\pi}{2}) = 1 \Rightarrow C = 1 \\
	\textbf{L{\"o}sung}\quad & y(x) = \arcsin (e^{\frac{-x^2}{2}})
\end{split}
\end{equation*}

\subsubsection{Variation der Konstanten}

\begin{equation*}
	\textbf{Grundsatz:}\quad y(x) = y_\text{homo}(x) + y_p(x)
\end{equation*}

\begin{equation*}
\begin{split}
	& y' - y = 1,\ y(0) = 0 \\
	\text{homogener Ansatz}\quad & y' = y \\
	\textbf{konstante L{\"o}sungen}\quad & y(x) \equiv 0 \\
	\text{Trennung}\quad & \frac{dy}{y} = dx \Rightarrow \int\frac{dy}{y} = \int dx \Rightarrow \log|y| = x \\
	\textbf{homogene L{\"o}sung}\quad & y_\text{homo}(x) = Ae^x,\ A = e^C \in \mathbb{R} \\
	\text{partikul{\"a}rer Ansatz}\quad & y_p(x) = A(x)e^x \\
	\text{einsetzen}\quad & A'e^x + Ae^x - Ae^x = 1 \Rightarrow A' = e^{-x} \Rightarrow A(x) = \int e^{-x}\ dx = -e^{-x} \\
	\textbf{partikul{\"a}re L{\"o}sung}\quad & y_p(x) = -1 \\
	\textbf{L{\"o}sung}\quad & y(x) = Ae^x - 1\ \text{mit Anfangsbedingung}\ A = 1 \\
	& \Rightarrow y(x) = e^x - 1
\end{split}
\end{equation*}

%\subsubsection{Substitution}
%
%\begin{equation*}
%\begin{split}
%	& y' = h(\frac{y}{x})\ \text{ersetzt durch}\ z(x) = \frac{y(x)}{x} \Leftrightarrow y(x) = xz(x) \\
%	& \Rightarrow	y' = z + xz'
%\end{split}
%\end{equation*}

\subsubsection{Euler-Ansatz}

\begin{equation*}
\begin{split}
	& y'' - 2y' - 8y = 0,\ y(1) = 1, y'(1) = 0 \\
	\text{Euler-Ansatz}\quad & y(x) = e^{\lambda x} \\
	\text{einsetzen}\quad & \lambda^2 e^{\lambda x} - 2\lambda e^{\lambda x} - 8e^{\lambda x} = 0 \\
	\textbf{charakt. Polynom}\quad & \lambda^2 - 2\lambda - 8 = (\lambda - 4)(\lambda + 2) = 0 \\
	\text{Nullstellen}\quad & 4, -2 \\
	\textbf{allgemeine L{\"o}sung}\quad & y(x) = Ae^{4x} + Be^{-2x} \\
	\text{Anfangsbedingung gebrauchen}\quad & y(1) = Ae^4 + Be^{-2} = 1,\ y'(1) = 4Ae^4 - 2Be^{-2} = 0 \\
											& \Rightarrow A = \frac{1}{3}e^{-4}, B = \frac{2}{3}e^2 \\
	\textbf{L{\"o}sung}\quad & y(x) = \frac{1}{3}e^{4x-4} + \frac{2}{3}e^{2-2x}
\end{split}
\end{equation*}

\emph{Bemerkung:} Zu einer $m$-fachen Nullstelle $\lambda$ geh{\"o}ren die $m$ linear unabh{\"a}ngigen L{\"o}sungen $e^{\lambda x}$, $x\cdot e^{\lambda x}$, ... , $x^{m-1}\cdot e^{\lambda x}$. Zur $m$-fachen Nullstelle $\lambda = 0$ geh{\"o}ren die L{\"o}sungen $1$, $x$, ... , $x^{m-1}$. \\
 
\emph{Komplexe Nullstellen:} \\

\begin{equation*}
	x = \frac{-b \pm \sqrt{b^2-4ac}}{2a}
\end{equation*}

Ein komplexes Nullstellenpaar der Form $\alpha \pm \beta i$ liefert folgende homogene L{\"o}sung:
\begin{equation*}
	y(x)=e^{\alpha x}(C_1\cos(\beta x) + C_2\sin(\beta x))
\end{equation*}

\subsubsection{Direkter Ansatz}

\begin{equation*}
	\textbf{Grundsatz:}\quad y(x) = y_\text{homo}(x) + y_p(x)
\end{equation*}

\begin{table}[H]
\centering
\begin{tabular}{|l|l|l|}
\hline
\textbf{Inhomogener Term $b(x)$} & \textbf{Ansatz f{\"u}r $y_p(x)$}	& \textbf{zu bestimmen}		\\ \hline
Polynom				& $Ax^2 + Bx + C$			& $A$, $B$, $C$		\\ \hline
$c e^{k x}$ & $Ae^{kx}$					& $A$				\\ \hline
$c\sin(kx)$ oder $c\cos(kx)$ & $A\sin(kx) + B\cos(kx)$ & $A$, $B$ \\ \hline

\end{tabular}
\end{table}

\emph{Bemerkung:} Kommt der gew{\"a}hlte Ansatz schon in der homogenen L{\"o}sung vor, so multipliziert man den Ansatz einfach mit $x$.

\begin{equation*}
\begin{split}
	& y'' - y' + \frac{1}{4}y = \cos(x) \\
	\text{homogener}\quad & y'' + y' + \frac{1}{4}y = 0 \\
	\text{Euler-Ansatz anwenden}\quad & \lambda^2 + \lambda + \frac{1}{4} = (\lambda + \frac{1}{2})^2 = 0 \\
	\textbf{homogene L{\"o}sung}\quad &\Rightarrow y_\text{homo}(x) = Ae^{-\frac{x}{2}} + Bx \cdot e^{-\frac{x}{2}} \\
	\text{Ansatz w{\"a}hlen}\quad & y_p(x) = a\cos(x) + b\sin(x) \\
							  & \Rightarrow y_p'(x) = -a\sin(x) + b\cos(x),\  y_p''(x) = -a\cos(x) -b \sin(x) \\
	\text{Einsetzen}\quad & (-a + b + \frac{a}{4})\cos(x) + (-b -a + \frac{1}{4}b)\sin(x) = \cos(x) \\
	\text{Koeffizientenvergleich}\quad & -\frac{3}{4}a + b = 1,\ -a-\frac{3}{4}b = 0 \\
	\textbf{partikul{\"a}re L{\"o}sung}\quad & y_p(x) = -\frac{12}{25}\cos(x) + \frac{16}{25}\sin(x) \\
	\textbf{L{\"o}sung}\quad & y(x) = Ae^{-\frac{x}{2}} + Bx \cdot e^{-\frac{x}{2}} -\frac{12}{25}\cos(x) + \frac{16}{25}\sin(x)
\end{split}
\end{equation*}

\section{Vektorfelder}

\subsection{Differenzial}

\begin{equation*}
	df = \begin{pmatrix}
		\frac{\partial f_1}{\partial x_1} & ... & \frac{\partial f_1}{\partial x_n} \\
		... & ... & ... \\
		\frac{\partial f_m}{\partial x_1} & ... & \frac{\partial f_m}{\partial x_n}
	\end{pmatrix}
\end{equation*}

\subsection{Gradient}

\begin{equation*}
	\text{grad}(f)=\nabla f=
	\begin{pmatrix}
		\frac{\partial f}{\partial x_1}\\
		...\\
		\frac{\partial f}{\partial x_n}\\
	\end{pmatrix}
\end{equation*}

Der Gradient zeigt in die Richtung der maximalen Zuwachsrate von $f$ und seine L{\"a}nge ist gleich der maximalen {\"a}nderung von $f$.

\emph{Bemerkung:} $f:\Omega \subset \mathbb{R}^n \mapsto \mathbb{R}$

\subsection{Hessematrix}

\begin{equation*}
	\text{Hess}(f)=
	\begin{pmatrix}
		\frac{\partial^2 f}{\partial^2x_1^2} & ... & \frac{\partial^2 f}{\partial x_1 \partial x_n}\\
		...&...&...\\
		\frac{\partial^2 f}{\partial x_n \partial x_1} & ... & \frac{\partial^2 f}{\partial x_n^2}
	\end{pmatrix}
\end{equation*}

Falls a hat in x0 nur positive eigenwerte dann ist es eine maximalstelle, falls sie hat nur negative eingewerte dann ist es eine minimalstelle, falls sie hat beide dann ist es ein sattelpunkt.

\subsection{Rotation}

\begin{equation*}
	\text{In}\ \mathbb{R}^3:\ \text{rot}(\vec{v})=\nabla\times \vec{v} =
	\begin{pmatrix}
		\frac{\partial v_3}{\partial y} - \frac{\partial v_2}{\partial z}\\
		\frac{\partial v_1}{\partial z} - \frac{\partial v_3}{\partial x}\\
		\frac{\partial v_2}{\partial x} - \frac{\partial v_1}{\partial y}
	\end{pmatrix},\ \text{in}\ \mathbb{R}^2:\ \text{rot}(\vec{v}) = \frac{\partial v_2}{\partial x} - \frac{\partial v_1}{\partial y}
\end{equation*}

\emph{Bemerkung:} Falls $\text{rot}(\vec{v})=0$, dann ist $\vec{v}$ konservativ (Potenzialfeld).

\subsection{Divergenz}
\begin{equation*}
	\text{div}(v)= \frac{\partial v_1}{\partial x} + \frac{\partial v_2}{\partial y} + ... 
\end{equation*}

\subsection{Potenzialfeld}

Ein Potenzialfeld ist konservativ. Das Potenzial $\Phi$ eines Potenzialfeldes ist gleich:

\begin{equation*}
	\nabla \Phi = \vec{v}
\end{equation*}

F{\"u}r ein Potenzialfeld gilt $\text{rot}(\vec{v})=0$ und es erf{\"u}llt die \textbf{Integrabilit{\"a}tsbedinungen}:

\begin{equation*}
	\frac{\partial v_i}{\partial x_j}=\frac{\partial v_j}{\partial x_i},\forall i \neq j
\end{equation*}

\section{Wegintegral}

\subsection{Standardmethode}

\begin{equation*}
	\textbf{Grundsatz:}\quad \int_\gamma \vec{v}\cdot d\vec{s} := \int_a^b \vec{v}(\vec{\gamma}(t)) \cdot \dot\vec{\gamma}(t)\ dt
\end{equation*}

\begin{equation*}
\begin{split}
	& \vec{v} = \binom{y}{0},\ \gamma:[0, 2\pi] \mapsto \mathbb{R}^2,\ t \mapsto \binom{t -\sin(t)}{1-\cos(t)} \\
	\text{parametrisieren}\quad & \text{hier bereits gegeben} \\
	\text{$\gamma$ ableiten}\quad & \dot\gamma = \binom{1-\cos(t)}{\sin(t))} \\
	\text{in Formel einsetzen}\quad & \int_\gamma \vec{v} \cdot d\vec{s} = \int_0^{2\pi} \binom{1-\cos(t)}{0}\cdot\binom{1-\cos(t)}{\sin(t)}\ dt \\
	&= \int_0^{2\pi} (1-\cos(t))^2\ dt = \int_0^{2\pi} (1-2\cos(t)+\cos^2(t))\ dt \\
	\textbf{L{\"o}sung}\quad & 2\pi - 0 + \pi = 3\pi
\end{split}
\end{equation*}

\subsection{In Potenzialfeldern}

\paragraph{Anforderung:} Das Vektorfeld $\vec{v}$ ist \textbf{konservativ}(= Potenzialfeld, der Weg ist egal). Es existiert ein Potenzial.

\begin{equation*}
	\textbf{Grundsatz:}\quad\int_\gamma \vec{v} \cdot d\vec{s} = \Phi(\text{Ende}) - \Phi(\text{Anfang})
\end{equation*}

\begin{equation*}
\begin{split}
	& \vec{v} = \binom{e^{xy}(1 + xy)}{e^{xy}x^2},\ \text{Kreisbogen von } (1,0)\ \text{nach}\ (-1,0) \\
	\text{gleichsetzen:}\quad & \vec{v} = \binom{e^{xy}(1 + xy)}{e^{xy}x^2} \overset{!}{=} \binom{\frac{\partial\Phi}{\partial x}}{\frac{\partial\Phi}{\partial y}} = \nabla\Phi \\
	& \frac{\partial\Phi}{\partial y} = e^{xy}x^2 \Rightarrow \Phi = \int e^{xy}x^2\ dy = xe^{xy} + C(x) \\
	\text{ableiten:}\quad & \frac{\partial\Phi}{\partial x} = e^{xy} + xye^{xy} + C' \overset{!}{=} e^{xy} + xye^{xy} \\
	& \Rightarrow C' = 0 \Rightarrow C = \text{const.} \\
	\textbf{Potenzial:}\quad & \Phi = xe^{xy} + \text{const.} \\
	\textbf{L{\"o}sung:}\quad & \int_\gamma \vec{v} \cdot d\vec{s} = \Phi(-1,0) - \Phi(1,0) = -1 + C -1 - C = 2
\end{split}
\end{equation*}

\subsection{Satz von Green}

\paragraph{Anforderung:} Der Rand muss im positiven mathematischen Sinn umlaufen werden (d.h. im Gegenuhrzeigersinn)

\begin{equation*}
	\textbf{Grundsatz:}\quad\int_{\gamma = \partial C} \vec{v} \cdot d\vec{s} = \int_C (\frac{\partial v_2}{\partial x}-\frac{\partial v_1}{\partial y})\ dxdy
\end{equation*}

\begin{equation*}
\begin{split}
	& \vec{v} = \binom{x+y}{y},\ \text{Kreisbogen mit Radius $1$ um $(0,0)$} \\
	\text{Rotation berechnen:}\quad & rot(\vec{v}) = \frac{\partial v_2}{\partial x}-\frac{\partial v_1}{\partial y} = 0 -1 = -1 \\
	\text{Normalbereich:}\quad & E = \{(x,y) \in \mathbb{R}^2 | x^2 + y^2 \leq 1 \} \\
	\text{in Formel einsetzen:}\quad & \int_\gamma \vec{v} \cdot d\vec{s} = \int_E -1\ dxdy = -\mu(E) = -\pi
\end{split}
\end{equation*}

\subsection{Satz von Stokes}

\paragraph{Anforderung:} Einfacher in $\mathbb{R}^3$, der Rand muss im positiven mathematischen Sinn umlaufen werden (d.h. im Gegenuhrzeigersinn)

\begin{equation*}
	\textbf{Grundsatz:}\quad\int_{\gamma = \partial C} \vec{v} \cdot d\vec{s} = \int_C \text{rot}(\vec{v}) \cdot \vec{n}\ do
\end{equation*}

\begin{equation*}
\begin{split}
	& \vec{v} = \begin{pmatrix}
		y(z^2-x^2) \\ x(y^2 - z^2) \\ z(x^2 + y^2)
	\end{pmatrix},\ \text{Rand der oberen H{\"a}lfte der Einheitssph{\"a}re mit Radius $1$ um $(0,0, 0)$} \\
	\text{Rotation berechnen:}\quad & rot(\vec{v}) = \begin{pmatrix}
		2z(x+y) \\ 2z(y-x) \\ x^2 + y^2 - 2z^2
	\end{pmatrix} \\
	\textbf{Einheitssph{\"a}re parametrisieren:}\quad & \Phi(\theta, \phi) = \begin{pmatrix}
		\sin(\theta)\cos(\phi) \\ \sin(\theta)\sin(\phi) \\ \cos(\theta)
	\end{pmatrix} \\
	\text{Normalvektor berechnen:}\quad & \Phi_\theta = \begin{pmatrix}
		\cos(\theta)\cos(\phi) \\ \cos(\theta)\sin(\phi) \\ -\sin(\theta)
	\end{pmatrix},\ \Phi_\phi = \begin{pmatrix}
		-\sin(\theta)\sin(\phi) \\ \sin(\theta)\cos(\phi) \\ 0
	\end{pmatrix} \\
	& \Phi_\theta \times \Phi_\phi = \begin{pmatrix}
		\sin^2(\theta)\cos(\phi) \\ \sin^2(\theta)\sin(\phi) \\ \sin(\theta)\cos(\theta) \\
	\end{pmatrix} \\
	\textbf{Grundsatz anwenden:} \quad & \int_H \text{rot}(\vec{v}) \cdot \vec{n}\ do = \frac{\pi}{2}
\end{split}
\end{equation*}

\section{Fl{\"a}chenintegral}

\subsection{Koordinatentransformationen}

\subsubsection{Polarkoordinaten ($\mathbb{R}^2$)}

\begin{equation*}
	\textbf{Variablen:}\quad \begin{matrix}
		x = r\cos(\phi) \\ y = r\sin(\phi)
	\end{matrix} \quad \textbf{Volumenelement:}\quad dxdy = r\ drd\phi
\end{equation*}

\subsubsection{Elliptische Koordinaten ($\mathbb{R}^2$)}

\begin{equation*}
	\textbf{Variablen:}\quad \begin{matrix}
		x = ra\cos(\phi) \\ y = rb\sin(\phi)
	\end{matrix} \quad \textbf{Volumenelement:}\quad dxdy = abr\ drd\phi
\end{equation*}

\subsubsection{Zylinderkoordinaten ($\mathbb{R}^3$)}

\begin{equation*}
	\textbf{Variablen:}\quad \begin{matrix}
		x = r\cos(\phi) \\ y = r\sin(\phi) \\ z = z
	\end{matrix} \quad \textbf{Volumenelement:}\quad dxdydz = r\ drd\phi dz
\end{equation*}

\subsubsection{Kugelkoordinaten ($\mathbb{R}^3$)}

\begin{equation*}
	\textbf{Variablen:}\quad \begin{matrix}
		x = r\sin(\theta)\cos(\phi) \\ y = r\sin(\theta)\sin(\phi) \\ z = r\cos(\theta)
	\end{matrix} \quad \textbf{Volumenelement:}\quad dxdydz = r^2\ dr\ \sin(\theta)\ d\theta d\phi
\end{equation*}

\subsection{Normalbereich}

\begin{equation*}
\begin{split}
	\textbf{Grundsatz:}\quad & \Omega = \{(x,y) \in \mathbb{R}^2| a \leq x \leq b, f(x) \leq y \leq g(x)\} \\
	& \int_\Omega F\ d\mu = \int_a^b dx \int_{f(x)}^{g(x)} dy\ F(x,y)
\end{split}
\end{equation*}

\begin{equation*}
\begin{split}
	& \int_\Omega xy\ d\mu,\ \Omega = \{(x,y) \in \mathbb{R}^2| y \geq x^2, x \geq y^2 \} \\
	\text{als Normalbereich schreiben:}\quad & \Omega = \{(x,y) \in \mathbb{R}^2| 0 \leq x \leq 1, x^2 \leq y \leq \sqrt{x}\} \\
	\text{in Formel einsetzen:}\quad & \int_\Omega xy\ d\mu = \int_0^1 dx\int_{x^2}^{\sqrt{x}} dyxy = \int_0^1 dx\ x \Big[\frac{y^2}{2}\Big]_{x^2}^{\sqrt{x}} \\
	& = \int_0^1 \Big(\frac{x^2}{2}-\frac{x^5}{2}\Big)dx = \frac{1}{12} \\
\end{split}
\end{equation*}

\emph{Bemerkung:} Soll nur die Fl{\"a}che ausgerechnet werden, so w{\"a}hle $F(x,y) = 1$. Werden Polarkoordinaten benutzt, so w{\"a}hle $F(r, \phi) = r$.

\subsection{Satz von Green}

\paragraph{Anforderung:} Der Rand muss im positiven mathematischen Sinn umlaufen werden (d.h. im Gegenuhrzeigersinn)

\begin{equation*}
	\textbf{Grundsatz:}\quad \mu(C) = \int_{\gamma = \partial C} \vec{v} \cdot d\vec{s}\text{, falls}\ rot(\vec{v}) = 1
\end{equation*}

\begin{equation*}
\begin{split}
	& \text{Fl{\"a}cheninhalt der Ellipse $E$, berandet durch}\ x = a\cos(\theta),\ y = b\sin(\theta) \\
	\text{Rand parametrisieren:}\quad & \gamma: [0, 2\pi] \mapsto \mathbb{R}^2,\ \theta \mapsto \binom{a\cos(\theta)}{b\sin(\theta)} \\
	\text{Vektorfeld ausw{\"a}hlen:}\quad & \vec{v}_1 = \binom{0}{x}\ \text{oder}\ \vec{v}_2 = \binom{-y}{0} \\
	\textbf{Wegintegral ausrechnen:}\quad & \mu(E) = \pi ab
\end{split}
\end{equation*}

\section{Oberfl{\"a}chenintegral}

\begin{minipage}[c]{0.5\textwidth}
\centering
\includegraphics[width=\linewidth,keepaspectratio=true]{images/fluss}
\end{minipage}
%
\begin{minipage}[c]{0.5\textwidth}
\begin{equation*}
\begin{split}
	\text{gegeben:}\quad & \vec{F}(x,y,z) = \begin{pmatrix}
		\frac{x^2}{2} \\ -xy \\ x^2 + 3z^2 - 3
	\end{pmatrix} \\
	\text{gesucht:}\quad & \text{Fluss durch die Mantelfl{\"a}che des Kegels} \\
	& \text{(von Innen nach Aussen)} \\
	\textbf{Vorgehen:}\quad & \text{Fluss durch den ganzen Kegel} \\
	& \text{mit \textbf{Satz von Gauss} berechnen} \\
	& \text{Fluss durch Deckel} \\
	& \text{mit \textbf{Standardmethode} berechnen}
\end{split}
\end{equation*}
\end{minipage}

\subsection{Standardmethode}

\begin{equation*}
	\textbf{Grundsatz:}\quad\int_{\partial V} \vec{v}\cdot\vec{n}\ do
\end{equation*}

\begin{equation*}
\begin{split}
	\text{Normalvektor:}\quad & \vec{n} = \begin{pmatrix}
		0 \\ 0 \\ 1
	\end{pmatrix} \\
	\text{Vektorfeld anpassen:}\quad & z = 1 \Rightarrow \vec{F} = \begin{pmatrix}
		\frac{x^2}{2} \\ -xy \\ x^2
	\end{pmatrix} \\
	\textbf{Grundsatz anwenden:}\quad & \iint_D \begin{pmatrix}
		0 \\ 0 \\ 1
	\end{pmatrix}\begin{pmatrix}
		\frac{x^2}{2} \\ -xy \\ x^2
	\end{pmatrix} dxdy = \iint_D x^2\ dxdy \\
	\textbf{Koordinatentransformation:}\quad & \int_0^{2\pi} \int_0^1 r^3\cos(\phi)\ drd\phi = \frac{\pi}{4}
\end{split}
\end{equation*}

\subsection{Satz von Gauss}

\begin{equation*}
	\textbf{Grundsatz:}\quad\int_{\partial V} \vec{v}\cdot\vec{n}\ do = \int_V \text{div}(\vec{v})\ d\mu
\end{equation*}
wobei $\vec{n}$ die nach aussen gerichtete Normale l{\"a}ngs $\partial V$ bezeichnet.

\begin{equation*}
\begin{split}
	\text{Divergenz berechnen:}\quad & \text{div}(\vec{F}) = 6z \\
	\textbf{Grundsatz anwenden:}\quad & \int_{-1}^1 dz \int_0^{2\pi} d\phi \int_0^{\frac{z+1}{2}} 6zr\ dr = 2\pi
\end{split}
\end{equation*}

\emph{Bemerkung:} In diesem Beispiel wurden zylindrische Koordinaten benutzt.

\subsection{Satz von Stokes}

\paragraph{Anforderung:} Einfacher in $\mathbb{R}^3$, der Rand muss im positiven mathematischen Sinn umlaufen werden (d.h. im Gegenuhrzeigersinn)

\begin{equation*}
	\textbf{Grundsatz:}\quad\int_{\gamma = \partial C} \vec{v} \cdot d\vec{s} = \int_C \text{rot}(\vec{v}) \cdot \vec{n}\ do
\end{equation*}

\section{Kurvendiskussion}

\begin{table}[H]
\centering
\begin{tabular}{|c|c|c|}
\hline
Extremalstelle & $f'(x) =  0 \land f''(x) \neq 0$ \\ \hline
Minimalstelle & $f'(x) = 0 \land f''(x) > 0$ \\ \hline
Maximalstelle & $f'(x) = 0 \land f''(x) < 0$ \\ \hline
Wendepunkt & $f''(x) = 0 \land f'''(x) \neq 0$ \\ \hline
Sattelpunkt & $f'(x) = 0 \land f''(x) = 0 \land f'''(x) \neq 0$ \\ \hline
\end{tabular}
\end{table}

\begin{description}[labelindent=16pt,style=multiline,leftmargin=6cm, noitemsep]
	\item[kritischer Punkt:] $p_0 \in \Omega$ f{\"u}r welchen $\text{rank}(df(p_0)) < \min\{m,n\}$ gilt
	\item[Kandidaten f{\"u}r Extrema:] $p_0 \in \Omega$ f{\"u}r welchen $df(p_0) = 0$ gilt
\end{description}

\subsection{Extremwertaufgaben ohne Nebenbedingungen}

\begin{enumerate}[noitemsep]
	\item Kandidaten f{\"u}r Extrema finden $df(x)=0$
	\item Bestimmung:
	\begin{enumerate}[noitemsep]
		\item $\text{Hess}(f)(p_0)$ positiv definit $\Rightarrow$ lokales Minimum
		\item $\text{Hess}(f)(p_0)$ negativ definit $\Rightarrow$ lokales Maximum
		\item $\text{Hess}(f)(p_0)$ indefinit $\Rightarrow$ Sattelpunkt
	\end{enumerate}
\end{enumerate}

\emph{Bemerkung:} Falls alle Eigenwerte von $A$ gr{\"o}sser als $0$ sind, dann ist $A$ \textbf{positiv definit}. Hat $A$ sowohl positive als auch negative Eigenwerte, so ist sie \textbf{indefinit}.

\subsection{Extremwertaufgaben mit Nebenbedingungen}

\begin{equation*}
\begin{split}
	\text{gegeben:} \quad & f = xyz\ \text{mit Nebenbedinung}\ x^2 + y^2 + z^2 = 1 \\
	\textbf{Lagrange-Bedingung:} \quad & L = f - \lambda g = xyz - \lambda(x^2 + y^2 + z^2 - 1) \\
	\text{kritische Punkte von $L$:} \quad & dL = 0 \\
	& \frac{\partial L}{\partial x} = 0 \Rightarrow \lambda = \frac{yz}{2x} \\
	& \frac{\partial L}{\partial y} = 0 \Rightarrow \lambda = \frac{xz}{2y} \\
	& \frac{\partial L}{\partial z} = 0 \Rightarrow \lambda = \frac{xy}{2z} \\
	\text{Lambdas gleichsetzen:} \quad & x^2 = y^2 = z^2 \land g \Rightarrow 3x^2 = 1 \Rightarrow x = \pm \frac{1}{\sqrt{3}} \\
	\textbf{Kandidaten:} \quad & \begin{pmatrix}
		\pm \frac{1}{\sqrt{3}}, & \pm \frac{1}{\sqrt{3}}, & \pm \frac{1}{\sqrt{3}}
	\end{pmatrix} \\
	\text{in $f$ einsetzen:} \quad & f\begin{pmatrix}
		\pm \frac{1}{\sqrt{3}}, & \pm \frac{1}{\sqrt{3}}, & \pm \frac{1}{\sqrt{3}}
	\end{pmatrix} = \pm \frac{1}{3\sqrt{3}}
\end{split}
\end{equation*}

\paragraph{Vorgehen um alle Kandidaten zu finden:}
\begin{enumerate}[noitemsep]
	\item Lagrange-Bedinung anwenden (m{\"u}ssen alle Nebenbedingung erf{\"u}llen)
	\item Kandidaten der Nebenbedingung \\
	falls $g$ differenzierbar: 
	\begin{enumerate}[noitemsep]
		\item nicht-regul{\"a}re Punkte finden mit $dg = 0$
		\item gefundene Punkte mit Nebenbedingung {\"u}berpr{\"u}fen
	\end{enumerate}
	falls $g$ nicht differenzierbar:
	\begin{enumerate}[noitemsep]
		\item nicht-regul{\"a}re Punkte der Teilst{\"u}cke des Randes
		\item Eckpunkte des Gebietes {\"u}berpr{\"u}fen
	\end{enumerate}
\end{enumerate}

\end{document}

\documentclass[11pt]{article}

\usepackage{amsmath}
\usepackage{amssymb}
\usepackage{array}
\usepackage{geometry}
\usepackage{enumitem}
\usepackage{float}
\usepackage{cancel}
\usepackage{graphicx}
\usepackage{textcomp}
\usepackage[labelformat=empty]{caption}

\geometry{
	a4paper,
 	left=20mm,
 	top=20mm,
}

\setlength{\parindent}{0pt}

\begin{document}

\section{Syntax}

\begin{description}
	\item[create\{IPLAYER\} player] creates an instance of IPLAYER and assigns it to player
	\item[player := create\{IPLAYER\}] same as create\{IPLAYER\} player
	\item[iplayer.play\_mp3(create\{MP3\_FILE\}.make)] initializes and passes an MP3-File to the feature
	\item[check not player.is\_playing end] Work just like pre-/postconditions, just that they're right in the body (assertion)
	\item[NATURAL] from 0 up
	\item[modulo] \textbackslash\textbackslash
	\item[integer division] 5 // 2 = 2
	\item[v ?= e] downcast, needs runtime checking
	\item[h@k] dictionary access (stands for h.at(k))
	\item[attached a\_node] synonym to a\_node /= Void
	\item[str $\sim$ str] same as str.is\_equal(str), just void-safe
\end{description}

\section{Semantics}

\begin{description}
	\item[Qualified call] explicitly lists the target object, e.g. $x.f(args)$
	\item[Feature export]
	\begin{itemize}
		\item Information hiding only applies to qualified calls. 
		\item Features exported to $NONE$ (a class inheriting from all classes) can not be accessed by clients. It may only be accessed within the defining class itself or its descendants.
		\item Creation procedures exported to $NONE$ may only be used as creation procedures by clients.
	\end{itemize}
\end{description}


\section{Reference types vs. expanded types}
\begin{description}
	\item[Reference types]
	contain the address (reference, or location of the object) \\
	$\Rightarrow$ they don't exist when we declare them (they are initally void)
	\begin{center}
	\includegraphics[width=250pt]{reference_type.png}
	\end{center}
	
	\item[Expanded types]
	points directly to the object \\
	$\Rightarrow$ they exist by just declaring them (they are never void) \\
	\begin{center}
	\includegraphics[width=250pt]{expanded_type.png}
	\end{center}
	
	\item[Object comparison] \mbox{} \\
	s1: STRING = "Teddy" \\
	s2: STRING = "Teddy" \\
	.. \\
	s1 = s2 (comparison) $\rightarrow$ FALSE: reference comparison on different objects\\
	s1 $\backsim s2$ (comparison) $\rightarrow$ TRUE: compares the content of two objects\\
	
\end{description}


\section{Containers}

\begin{description}
	\item[Hash Tables]
	\begin{itemize}
		\item Open Hashing: Every entry has a Linked List, collided keys are therefore just stored in the same list.
		\item Closed Hashing: In case of a collision, look for open spots somewhere around the index, where the collision took place (that's what Hash Tables in Eiffel use)
	\end{itemize}
 	\item[Tuples] Tuple [A,B,C] conforms to [A,B] and [A]
\end{description}

\section{Design by Contracts}

\begin{description}
	\item[Class invariant] Must be satisfied after creation and after the execution of any feature by any client (so affects \textbf{qualified} calls only, \textbf{unqualified} calls and calls to \textbf{non-exported} features may break the invariant) \\
		If inherited, may only be stronger or equally strong
	\item[Precondition] If inherited, may only be weaker or equally strong
	\item[Postcondition] If inherited, may only be stronger or equally strong
	\item[Loop Invariant] Boolean expression to determine whether the loop achieves its purpose. Needs to be $TRUE$ after initialization of the loop and preserved by the loop body (so has to be true after the last step as well).
	\item[Loop Variant] Integer expression to determine whether the loop will terminate. Has to be non-negative after initialization of the loop and decreased by at least one, while remaining non-negative, by any execution of the loop body.
\end{description}

\section{Agents}

\begin{description}
	\item[LIST]
	\begin{itemize}
		\item do\_all(action: PROCEDURE) \\ apply action to every item
		\item do\_if(action: PROCEDURE, test: FUNCTION) \\ apply action to every item that satisfies test
		\item there\_exists(test: FUNCTION): BOOLEAN \\ is test true for at least one element
		\item for\_all(test: FUNCTION) \\ is test true for all elements
	\end{itemize}
\end{description}


\end{document}

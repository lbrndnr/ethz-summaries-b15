\newpage
\section{Verilog}
	\lstset{language=Verilog}
	\lstset{basicstyle=\ttfamily\mdseries}
	\begin{multicols}{2}
	
	\subsection{Operatoren}
		\begin{tabular}{ll}
		\lstinline!+!, \lstinline!-!, \lstinline!*! & Arithmetic operators \\
		\lstinline*/ * &  Integer division (fractional part truncated) \\
		\lstinline*%* &  Modulo (takes sign of the first operand) \\
		\lstinline!**! &  Exponent \\
		\lstinline*-* & Negation (2's complement) \\
		\\
		\lstinline*~* & Bitwise NOT (1's complement) \\
		\lstinline*&* & Bitwise AND \\
		\lstinline*|* & Bitwise OR \\
		\lstinline*^* & Bitwise XOR \\
		\lstinline*^~*, \lstinline*~^* & Bitwise XNOR \\ 
		\\
		\lstinline*<<*, \lstinline*>>* & Logical shift (padding with 0) \\
		\lstinline*<<<*, \lstinline*>>>* & Arithmetic shift (padding with leftmost \\
		& bit when shifting right) \\ 
		\lstinline*?:* & Conditional \\
		\lstinline*{ a, b }* & Concatenation \\
		\lstinline*{ num { a }}* & Replication \\
		\end{tabular} 
	
	\subsection{Vergleiche}
		\begin{tabular}{ll}
		\lstinline*>*, \lstinline*<*, \lstinline*>=*, \lstinline*<=* & 	Relational operators (\lstinline*0*, \lstinline*1* or \lstinline*x*) \\
		\lstinline*==*, \lstinline*==* & Logical equality (\lstinline*0*, \lstinline*1* or \lstinline*x*) \\
		\lstinline*===*, \lstinline*!==* & Case equality (\lstinline*0* or \lstinline*1*) \\
		\end{tabular}
	
	
	\subsection{Konstanten}
		Angabe mit [\textless width in bits\textgreater]'\textless base\textgreater\textless number\textgreater \\
		\begin{tabular}{ll}
		\lstinline*'b* &(binär, 2) \\
		\lstinline*'o* &(oktal, 8) \\
		\lstinline*'d* &(dezimal, 10) \\
		\lstinline*'h* &(hexadezimal, 16)
		\end{tabular} \\
		z.B. \lstinline*-4'd3* (\lstinline*1101*), \lstinline*4'b11* (\lstinline*0011*), \lstinline*'h08FF* (16 bit hex)
		
	
	
	\subsection{Module}
	\lstset{basicstyle=\ttfamily\scriptsize\mdseries}
	In Verilog bestehen die Modelle aus Modulen, wobei jedes Modul ein Interface besitzt, welches die Inputs und Outputs definiert.
	Ein Modul kann dann instanziert werden, wobei ihm ein eindeutiger Name zugewiesen wird.
		\begin{lstlisting}[language=Verilog]
// modul definieren
module abc(input a,b, input [3:0] data, output out);
	wire x;
	assign x = (!a || b) ^ a;
	assign out = x ;
endmodule

// module instanzieren
module main(..)
	abc A1(a,x,data,myvar);
endmodule
		\end{lstlisting}
	
	\vfill
	
	\subsection{always}
	\lstset{basicstyle=\ttfamily\normalsize\mdseries}
	Mit \lstinline+always+ lassen sich Endlosschleifen ausdrücken. Zusätzlich
	kann die Ausführung auf positive/negative Flanken einer bestimmten Variable
	eingeschränkt werden.
	\lstset{basicstyle=\ttfamily\scriptsize\mdseries}
		\begin{lstlisting}[language=Verilog]
always @([posedge | negedge] var) begin .. end
		\end{lstlisting}
	
		
	\subsection*{Zeitauflösung/Delays}
	\lstset{basicstyle=\ttfamily\normalsize\mdseries}
	\begin{tabular}{lp{7.3cm}}
		\lstinline+'timescale <Zeiteinheit>/<Aufloesung>+ &
	\end{tabular}
	Wird im Header eines Moduls angegeben. z.B. \lstinline+1ns/100ps+ bedeutet dass der Delay-Operator \lstinline+#1+ für 1ns verzögert und die Simulation in 100ps Zeitschritten läuft. 
			
	\subsection*{Zuweisungsoperatoren}
	\lstset{basicstyle=\ttfamily\normalsize\mdseries}
	\begin{tabular}{lp{6.5cm}}
		\lstinline+a = b+   &   \emph{blocking assignment}: Der Wert der Zielvariable wird sofort aktualisiert. \\
		\lstinline+a <= b+   &   \emph{non-blocking assignment}: Die rechte Seite wird	
	sofort ausgewertet, die Zuweisung erfolgt jedoch erst im nächsten Zeitschritt (nicht Flanke/Takt!), was eine parallele
	Ausführung solcher Befehle ermöglicht. Die NBA's bewirken keine Verzögerung im Timingdiagramm (ausser wenn die Auflösung
	gleich der Zeiteinheit ist, was nicht sehr sinnvoll ist). \\
		\lstinline+assign a = b+ & \emph{continuous assignment}: Weist der Zielvariable sofort den Wert des Ausdrucks zu, wenn immer dieser ändert. Ausserhalb von always und initial Blöcken!
	\end{tabular}
	
	
	\subsection*{Weitere Konstrukte}
	\lstset{basicstyle=\ttfamily\normalsize\mdseries}
	\begin{tabular}{lp{5cm}}
		\lstinline+reg a,b;+   &   Lokale 1-Bit Variablen (Register) \\
		\lstinline+reg [3:0] d;+   &   Lokale 4-Bit Variablen \\
		\lstinline+wire a+   &   Draht, verbindet z.B. Ein-/Ausgänge von Modulen \\
		\lstinline+assign #3 a = b & c+   &   Zuweisung (Delay: 3 Einheiten) \\
	\end{tabular}
	
	
	\end{multicols}
	
	
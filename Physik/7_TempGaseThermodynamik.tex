\documentclass[7pt]{article}
\usepackage{amsmath}
\usepackage{graphicx}
\usepackage{caption}
\usepackage[landscape]{geometry}
\usepackage{multicol}
\usepackage{amssymb}
\usepackage{geometry}
\geometry {
 a4paper,
 total={285mm,185mm},
 left=10mm,
 top=10mm,
}

\setcounter{section}{6}
\setlength\columnseprule{0.5pt}
\setlength{\parindent}{0pt}
\graphicspath{ {images/} }


\begin{document}
\begin{multicols*}{3}

\section{Temperatur, Gase,    Thermodynamik}
\subsection{Atome}
Der massenzahl: A = Z + N wo Z ist die anzahl protonen und electronen, N ist die Anzahl Neutronen.
\subsection{Temperatur / Gasthermometer}
\paragraph{Druck von ein Gas}
\begin{equation}
	p=\frac{F}{A}
\end{equation}
F ist die Kraft und A die Flache.\\
pascal:$NM^{-2}$, 1 atm = $1,0125 * 10^5$ Pa, 1 bar = $10^5$Pa
\subsection{Absolute Temperatur / Kelvin-Skala}

\subsection{Ideale Gase}
\begin{equation}
	pV=NkT
\end{equation}
wo k die boltzman konstante, N die Anzahl der gasmolekule, T die temperatur(K), V die Volume($m^3$) und p die Druck(Pa).
\subsection{W{\"a}rmeenergie und W{\"a}rmekapazit{\"a}t}
Warmekapazitat C:
\begin{equation}
	C=\frac{\Delta Q}{\Delta T}
\end{equation}
Wo Q ist die energie benotigt um den korper temperatur um T tauschen. Mann kann auch die warmkapazitat pro masse oder pro mol definieren. Wenn $\Delta T$ nicht zu gross ist, ist C eine konstante.
\paragraph{Warmkapazitat einer idealen einatomiges Gasses bei konstanten Volume V}
\begin{equation}
	C_V=\frac{3}{2}Nk\quad [C_V] = \frac{J}{K}
\end{equation}
\paragraph{Warmkapazitat einer idealen einatomiges Gasses bei konstanten Druck p}
\begin{equation}
	C_p=\frac{5}{2}Nk = C_V+Nk
\end{equation}
fur zweiatomogis $\frac{7}{2}$\\
Die warmkapazitat von meisten korper ist nur abhangig von Anzahl Molen:
\begin{equation}
	c\approx 25\frac{J}{mol*K}
\end{equation}
\subsubsection{Mischtemperatur}
\begin{equation}
	C_1(T_{ende}-T_0)=C_2(T_{1}-T_{ende})
\end{equation}


\subsection{Latente W{\"a}rme}
Energie gebraucht fur ein Phasenubergang
\begin{equation}
	Q=mL
\end{equation}
Wo L eine konstante specifik zu jede Substanz ist.

\subsection{W{\"a}rmestrahlung}
Jede korper emittiert und absorbiert strahlung. Wenn er warmer als sein Umgebung ist dann emittiert er mehr als er absorbiert und vis versa, bis thermische gleichgewicht.
\paragraph{Die Austrahlung}
\begin{equation}
	S(T)=\epsilon\sigma T^4
\end{equation}
Wobei $\sigma$ eine konstante ist und die unitat von S(T)=$\frac{J}{sm^2}$ und $0\geq\epsilon \leq 1$ 1 bei idealen Fall.

\subsection{Erster Haupsatz - Thermodynamik}
In einer geschlosennes System wird nach ein Zeit ein termischer Gleichgewicht erreicht.
\paragraph{Die innere Energie U}
Ist die gesamte Energie in eine System. U hangt nur von anfang und endzustand. $U = U_e-U_a$.
\paragraph{Innere Energie des idealen Gasses}
\begin{equation}
	U=\frac{3}{2}NkT=\frac{3}{2}pV
\end{equation}
\begin{equation}
	dU=dQ+dW
\end{equation}

\subsection{Mechanische Arbeit eines expandierenden Gases}
Die energie ist von Gas geleistet
\begin{equation}
	dW=-Fdx=-(pA)dx=-pdV
\end{equation}
\begin{equation}
	W=-\int^{V_e}_{V_a}pdV=-p(V_e-V_a)
\end{equation}
bei konstanten Druck
\subsubsection{Isotherme expansion}
Temperatur des gasses bleibt konstant und energie kommt von aussen.
\begin{equation}
	Q=\int dQ=-\int dW=-W
\end{equation}
\subsubsection{Adiabatische Ausdehnung}
Keine warme wird dem Gas ausgetauscht $\Rightarrow dQ=0$, Temperatur des Gases wahrend der adiabatischen Expansion abnimmt. Bei der adiabatischen Expansion wird die im Gas gespeicherte Warmeenergie in mechanische Arbeit umgewandelt.
\begin{equation}
	pV^\gamma=konst
\end{equation}
\begin{equation}
	TV^{\gamma -1 }=konst
\end{equation}
$\gamma=\frac{5}{3}$, fur ein zweiatomiges $\gamma=\frac{7}{5}$

\subsection{Thermische Prozesse des idealen Gases}

\subsection{W{\"a}rmemaschine}
Warme ist in energie umgewandelt
\begin{equation}
	Q_{isotherm}=-W_{isotherm}=nRTln(\frac{V_2}{V_1})
\end{equation}
$V_1$ anfangs volum, n ist Anzahl Atomen.
\paragraph{Wirkungsgrad}
Der Wirkungsgrad einer W ̈armemaschine ist definiert als Verh ̈altnis der geleisteten Arbeit zur zugefu ̈hrten W ̈arme:
\begin{equation}
	\epsilon=\frac{|W|}{|Q_W|}=\frac{|Q_W|-|Q_K|}{|Q_W|}=1-\frac{Q_W}{Q_K}
\end{equation}
\paragraph{Leistungszahl}
In  ̈ahnlicher Weise ist die Leistungszahl einer W ̈armepumpe definiert als das Verh ̈altnis der W ̈arme, die dem kalten Reservoir entnommen wurde $(QK > 0)$, und der zugefu ̈hrten mechanischen Arbeit $(W > 0)$:
\begin{equation}
	c_L=\frac{Q_K}{W}
\end{equation}

\subsection{Zweiter Hauptsatz - Thermodynamik}


\end{multicols*}
\end{document}

















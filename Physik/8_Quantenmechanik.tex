\documentclass[7pt]{article}
\usepackage{amsmath}
\usepackage{graphicx}
\usepackage{caption}
\usepackage[landscape]{geometry}
\usepackage{multicol}
\usepackage{amssymb}
\usepackage{geometry}
 \geometry{
 a4paper,
 total={285mm,185mm},
 left=10mm,
 top=10mm,
 }

\setlength\columnseprule{0.5pt}
\setlength{\parindent}{0pt}
\graphicspath{ {images/} }


\begin{document}
\begin{multicols}{3}
	
\section{Quantenmechanik}

\paragraph{Balmer-Rydberg-Formel}

Für ein festes $m$ (z.B. $m = 2$) liefert die Formel eine Serie von Linien mit Wellenlängen, die sich nähern, wenn die Zahl $n$ zunimmt.

\begin{equation*}
	\frac{1}{\lambda} = R(\frac{1}{m^2}-\frac{1}{n^2})
\end{equation*}
wobei $m$ und $n$ positive ganze Zahlen sind. \newline

\paragraph{Frequenz einer Wellenlänge}

\begin{equation*}
	v = \frac{c}{\lambda} = Rc(\frac{1}{m^2} - \frac{1}{n^2})
\end{equation*}

\paragraph{Energie eines Atoms}

\begin{equation*}
	E = E_{kin} + E_{pot} = \frac{1}{2}m_e v_e^2 - \frac{1}{4\pi\varepsilon _0}\frac{e^2}{r}
\end{equation*}
Diese Gleichung entspricht der klassischen Mechanik (wenn sich das Elektron mit Radius $r$ um das Proton bewegt). \newline

Geht ein Atom von der Energie $E_n$ in die niedrigere Energie $E_m$ über, so ist die Frequenz $v$ des emittierten Lichts gleich
\begin{equation*}
	v = \frac{1}{h}(E_n - E_m)
\end{equation*}

\begin{equation*}
	E_n = -\frac{hcR}{n^2}
\end{equation*}

\paragraph{Elektron}

Der \textbf{Drehimpuls} ist ein ganzzahliges Vielfaches von $\hbar$ und somit auf bestimmte Werte beschränkt.
\begin{equation*}
	L = rp = rm_e v = \frac{nh}{2\pi} = n\hbar
\end{equation*}

Die max. kin. Energie, die ein Elektron nach dem Verlassen einer mit Licht bestrahlten Metalloberfläche haben kann, ist gleich
\begin{equation*}
	E_k = hv - A
\end{equation*}
wobei $A$ die Austrittsarbeit ist.

Die Energie eines einzelnen Elektrons ist gleich
\begin{equation*}
	E = hv = \frac{h\omega}{2\pi} \equiv \hbar\omega
\end{equation*}

Der Impuls eines einzelnen Elektrons ist gleich
\begin{equation*}
	p = \frac{h}{\lambda} = \frac{hk}{2\pi} \equiv \hbar k
\end{equation*}

\paragraph{Photon}

Die Energie eines einzelnen Photons ist gleich
\begin{equation*}
	E = hv = \frac{hc}{\lambda}
\end{equation*}

\end{multicols}
\end{document}